\documentclass[10pt,fleqn]{article} % Default font size and left-justified equations
\usepackage[%
    pdftitle={Modélisation systèmes multiphysiques : Modélisation linéaire et non linéaire},
    pdfauthor={Xavier Pessoles}]{hyperref}
    
\input{style/new_style}
\input{style/macros_SII}
\usepackage{multicol}
\usepackage{standalone}
\standaloneconfig{mode=buildnew}
\usepackage{siunitx}
\usepackage{wrapfig}
\fichetrue

%\fichefalse

%\proftrue
%\proffalse

\tdtrue
%\tdfalse

\courstrue
\coursfalse

\def\discipline{Sciences \\Industrielles de \\ l'Ingénieur}
\def\xxtete{Sciences Industrielles de l'Ingénieur}

\def\classe{PSI$\star$}
\def\xxnumpartie{\textsf{Cycle 01}}
\def\xxpartie{Modéliser le comportement linéaire et non linéaire des systèmes multiphysiques}


\def\xxnumchapitre{Chapitre 1 \vspace{.2cm}}
\def\xxchapitre{\hspace{.12cm} Modélisation multiphysique}


\def\xxtitreexo{\noindent Accéléromètre Grove IMU 9DOF}
\def\xxsourceexo{\hspace{.2cm} \footnotesize{Xavier Pessoles}}


\def\xxposongletx{2}
\def\xxposonglettext{1.45}
\def\xxposonglety{20}
%\def\xxonglet{Part. 1 -- Ch. 3}
\def\xxonglet{\textsf{Cycle 02}}

\def\xxactivite{Colle 1}
\def\xxauteur{\textsl{Xavier Pessoles}}

\def\xxcompetences{%
\textsl{%
\textbf{Savoirs et compétences :}\\
\textit{****}
%Les sources sont associées par un \emph{hacheur série}. La détermination des grandeurs électriques associées à ce montage permet de conclure vis à vis du cahier des charges.
%\noindent \textbf{Résoudre :} à partir des modèles retenus :
%\begin{itemize}[label=\ding{112},font=\color{ocre}] 
%\item choisir une méthode de résolution analytique, graphique, numérique;
%\item mettre en \oe{}uvre une méthode de résolution.
%\end{itemize}
%\begin{itemize}[label=\ding{112},font=\color{ocre}] 
%\item \textit{Rés -- C1.1 :} Loi entrée sortie géométrique et cinématique -- Fermeture géométrique.
%\end{itemize}
%
%\noindent \textit{Mod2 -- C4.1 :} Représentation par schéma-blocs.
}}

\def\xxfigures{
\includegraphics[width=.3\linewidth]{images/fig_01}
}%figues de la page de garde


\def\xxpied{%
Cycle 01 -- Modéliser le comportement des systèmes multiphysiques\\
\xxactivite%
}

\setcounter{secnumdepth}{5}
%---------------------------------------------------------------------------
%---------------------------------------------------------------------------


\begin{document}
%\chapterimage{png/Fond_Cin}
\input{style/new_pagegarde}
\vspace{4.5cm}
\pagestyle{fancy}
\thispagestyle{plain}


\def\columnseprulecolor{\color{ocre}}
\setlength{\columnseprule}{0.4pt} 

%\ifprof
%\else
\begin{multicols}{2}
%\fi
\subsection*{Matériel}
\begin{itemize}
\item Carte Arduino Mega
\item Shield Grove
\item Accélérmetre Grove IMU 9DOF V1
\end{itemize}

\subsection*{Logiciel}

Pour récupérer les données de l'accéléromètre, on utilise le protocole de communication I2C qui permet d'avoir accès à un réseau de données. En effet, toutes les valeurs lues par l'accéléromètre sont diffusées en réseau.

La bibliothèque Arduino permettant de travailler avec le protocole I2C est la libraire << Wire >>. Il semblerait que celle-ci ait plusieurs inconvénients :
\begin{itemize}
\item elle ne travaille qu'avec les broches A4/A5 (SDA/SCL), ce qui peut être problématique si ces broches sont utilisées pour d'autres fonctionnalités;
\item la même adresse I2C est utilisée pour différents capteurs. Cela peut être problématique, car parfois, par défaut, une même série de capteurs sont fabriqués avec la même adresse I2C. Il peut être alors nécessaire de la changer.
\end{itemize}

Une alternative à la bibliothèque Wire semble être la bibliothèque SoftwareI2C Library. \url{https://github.com/Seeed-Studio/Arduino_Software_I2C/archive/master.zip}

\subsection*{Un peu d'I2C}
\end{multicols}
\end{document}

\subparagraph{}\textit{}
\ifprof
\begin{corrige}~\\
\end{corrige}
\else
\fi

\begin{center}
\includegraphics[width=\linewidth]{images/img_04}
%\textit{}
\end{center}


\subparagraph{}\textit{}
\ifprof
\begin{corrige}~\\
\end{corrige}
\else
\fi

\subparagraph{}\textit{}
\ifprof
\begin{corrige}~\\
\end{corrige}
\else
\fi

\subparagraph{}\textit{}
\ifprof
\begin{corrige}~\\
\end{corrige}
\else
\fi

\subparagraph{}\textit{}
\ifprof
\begin{corrige}~\\
\end{corrige}
\else
\fi

\subparagraph{}\textit{}
\ifprof
\begin{corrige}~\\
\end{corrige}
\else
\fi

\subparagraph{}\textit{}
\ifprof
\begin{corrige}~\\
\end{corrige}
\else
\fi
\begin{center}
%\includegraphics[width=\linewidth]{images/fig_05}
\end{center}
